\documentclass[a4paper, 11pt]{article}
\usepackage[T1]{fontenc}
\usepackage[utf8]{inputenc}
\usepackage[norsk]{babel}
\usepackage{graphicx}
\usepackage[dvipsnames]{xcolor}
\usepackage{amsmath}
\usepackage{amsthm}
\usepackage{amssymb}
\usepackage{microtype}
\usepackage{array}
\usepackage{caption}
\usepackage{subcaption}
\usepackage[norsk]{babel}
\usepackage[all]{nowidow}
\usepackage{fancyhdr}
\usepackage{mathtools}
\usepackage{sectsty}
\usepackage{etoolbox}
\usepackage{cancel}
\usepackage{listings}
\usepackage{inconsolata}
\usepackage{tikz}
\usetikzlibrary{calc,arrows,patterns}
\usepackage{pgfplots}


% Unicode-inkompatibilitet mactex/texlive?
\DeclareUnicodeCharacter{00A0}{ }

% Få fancyhdr til å holde kjeft
\setlength{\headheight}{14pt} 

\newcommand{\tittel}{Øving 6}
\newcommand{\fag}{Diskret Matematikk}
\newcommand{\fagkode}{TMA4140}
\newcommand{\forfatter}{Steffen Haug}

\renewcommand{\qedsymbol}{$\themecolor{\blacksquare}$}

\newcommand{\ceq}[1]{\stackrel{\mathclap{\tiny\mbox{#1}}}{=}}
\newcommand{\R}{\mathbb{R}}
\newcommand{\Z}{\mathbb{Z}}
\newcommand{\Mod}{\;\text{\bf mod}\;}
\newcommand{\Modulo}{\text{mod }}
\newcommand{\deloppg}[1]{\vspace{1mm}\noindent \textbf{\tt \themecolor{#1:}}}
\newcommand{\dint}{\int\displaylimits}

\newcommand{\themeshade}{Mahogany}
\newcommand{\themecolor}[1]{\textcolor{\themeshade}{#1}}
\sectionfont{\color{\themeshade}}

\def\dul#1{\underline{\underline{#1}}}

% Forside
\newcommand*{\titleTH}{\begingroup 
    \raggedleft
    \vspace*{\baselineskip}
    {\Large \forfatter}\\[0.167\textheight]
    {\LARGE\bfseries \tittel}\\[\baselineskip]
    {\themecolor{\Huge \fag}}\\[\baselineskip]
    {\Large \textit{\fagkode}}\par
    \vfill
\endgroup}

\pgfplotsset{compat=1.15}

\begin{document}


\lstdefinelanguage{Lie}
{
  % list of keywords
  morekeywords={
    include,
    if,
    let,
    fn,
    lambda,
  },
  sensitive=true,       % keywords are not case-sensitive
  morecomment=[l]{;},  % l is for line comment
  morestring=[b]"       % double quoted strings
}

\lstdefinestyle{themecode}{
    keywordstyle=\bfseries\themecolor,
}

\lstset{
    numbers=left,
    basicstyle=\ttfamily,
    style=themecode,
    extendedchars=true,
    literate={λ}{{\themecolor{$\lambda$}}}1
             {*}{{$*$}}1
             {+}{{$+$}}1
             {-}{{$-$}}1
             {/}{{$/$}}1
             {=}{{$=$}}1
             {->}{{$\rightarrow$}}2,
}

% Front Page
\pagestyle{empty}
\titleTH
\newpage


\section*{Oppgåver til seksjon 6.5} % 6 14 30 54
Eg viser ikkje utrekning for \(C(n, r)\). Verdiane er rekna slik:
\begin{lstlisting}[language=Lie]
(fn zero? [x] (= x 0))

(fn decr [x] (- x 1))

(fn fac [n]
  (if (zero? n) 1
  (* n (fac (decr n)))))

(fn ncr [n r]
  (div (fac n) (* (fac r) (fac (- n r)))))
\end{lstlisting}


\subsection*{Oppgåve 6}
Frå Teorem 2: Å velje 5 element frå ei mengd av 3, med tilbakelegg, kan gjerast på
\begin{align*}
    C(n + r - 1, r) = C(3 + 5 - 1, 5) = \dul{21}
\end{align*}
forskjellige måtar.


\subsection*{Oppgåve 14}
Skal finne kor mange løysingar som eksisterer for \(x_1 + x_2 + x_3 + x_4 = 17\), der \(x_1, x_2, x_3, x_4\)
er positive heiltal. Merk at dette er det same som å velja 17 element frå ei mangd med 4 (med tilbakelegg),
slik at vi har \(x_1\) element av type ein, \(x_2\) element av type to, og so vidare. Antal løysingar er
dermed talet på 17-kombinasjonar av ei mengde med 3 element, med tilbakelegg.
\begin{align*}
    C(3 + 17 - 1, 17) = \dul{171}
\end{align*}


\newpage
\subsection*{Oppgåve 30}
Skal finne kor mange forskjellige ord som kan stavast med bokstavar frå ordet {``MISSISSIPPI''}.
Skal med andre ord finne antal permutasjonar av elleve bokstavar, med omsyn til at nokre av dei er like.
Ordet har:

1 M, som kan plasserast på \(C(11,1)\) måtar, og etterlét 10 plasseringar

4 I, som kan plasserast på \(C(10,4)\) måtar, og etterlét 6 plasseringar

4 S, som kan plasserast på \(C(6, 4)\) måtar, og etterlét 2 plasseringar

2 P, som kan plasserast på \(C(2, 2)\) måtar

\noindent Som gjer
\begin{align*}
    \text{antal ord}&= C(11,1)C(10,4)C(6,4)C(2,2)  \\
    &= \text{\tt (foldl * [(ncr 11 1) (ncr 10 4) (ncr 6 4) (ncr 2 2)])} \\
    &= \dul{34650}
\end{align*}


\subsection*{Oppgåve 34}
Skal finne kor mange ord med fem eller fleire bokstavar som kan lagast av bokstavar frå ordet ``SEERESS''.
Vi har sju bokstavar, med frekvensar S: 3, E: 3, R: 1

\vspace{3mm}
\noindent Vi kan lage:

\(C(7,3)C(4,3)C(1,1) = 140\) ord med 7 bokstavar

\noindent 6 bokstavar:

\(C(6,3)C(3,3) = 20\) ord utan R

\(2 \cdot C(6,3)C(3,2)C(1,1) = 60\) ord utan ein S eller E

\noindent 5 bokstavar:

\(2 \cdot C(5,3)C(2,2) = 20\) ord utan R, og E eller S

\(C(5,2)C(3,2)C(1,1) = 30\) ord utan ein S og ein E

\(2 \cdot C(5,1)C(4,3)C(1,1) = 40\) ord utan to S, eller to E

\vspace{3mm}
\noindent Totalt 140 + 60 + 20 + 30 + 40 = \dul{290 ord}


\newpage
\section*{Oppgåver til seksjon 8.1}
\subsection*{Oppgåve 11}
\deloppg{a} Skal setje opp ein rekurrensrelasjon som beskriv antal måtar å gå opp \(n\) trappetrinn,
dersom ein kan gå eitt eller to trappetrinn i gongen.

Dersom ein frå eit vilkårleg trinn, \(a_n\), går opp to trinn,
kan ein ``fullføre'' trappa på alle måtane som er mogleg frå dette trinnet, \(a_{n-2}\),
Dersom ein går eitt trinn kan ein fullføre trappa på \(a_{n-1}\) måtar. Dette gjev
\begin{align*}
    a_n = a_{n-1}+a_{n-2}
\end{align*}

\deloppg{b} Der er éin måte å gå ei trapp med eitt trinn(!), og to måtar å gå ei trapp med to trinn (1-1 eller 2).
\begin{align*}
    a_1 = 1, \quad a_2 = 2
\end{align*}

\deloppg{c} Skal finne kor mange måtar der er å gå ei trapp på 8 trinn.
\begin{lstlisting}[language=Lie]
(fn trapp [n]
  (if (= n 1) 1
  (if (= n 2) 2
  (+ (trapp (- n 1)) (trapp (- n 2))))))
\end{lstlisting}
{\tt >>> (trapp 8) \(\rightarrow\) 34}


\subsection*{Oppgåve 20}
\deloppg{a} Skal finne rekurrensrelasjon for kor mange måtar ein bussjåfør kan betale bompengar (\(n\) cent) med berre
{\em nickels} (5 cent) og {\em dimes} (10 cent).

Merk at bussjåføren kun kan betale bompengar slik at \(n \equiv 0 \;(\Modulo 5)\).
For å betale \(n\) cent kan sjåføren anten putte ein {\em nickel} i automaten, og
ha \(n - 5\) cent att å betale, eller putte ein {\em dime} i automaten, og ha \(n - 10\)
cent att å betale. Dette gjev
\begin{align*}
    a_n = a_{n - 5} + a_{n - 10}, \quad a_5 = 1, \; a_{10} = 2
\end{align*}
måtar å betale bompengar på \(n\) cent.

\deloppg{b} Skal finne kor mange måtar det er mogleg å betale bompengar på 45 cent
\begin{lstlisting}[language=Lie]
(fn toll [n]
  (if (= n 5) 1
  (if (= n 10) 2
  (+ (toll (- n 5)) (toll (- n 10))))))
\end{lstlisting}
{\tt >>> (toll 45) \(\rightarrow\) 55}


\section*{Oppgåver til seksjon 8.2} % 3c 3d 3e 3g 6 11 42
\subsection*{Oppgåve 3}
Skal løyse rekurrensrelasjonane, saman med initialbetingelsane.

\deloppg{c} \(a_n = 5a_{n-1} - 6a_{n-2}, \quad n \geq 2, \quad a_0 = 1, \; a_1 = 0\)

\noindent Karakteristisk likning: \(r^2 - 5r + 6 \rightarrow r_1=2, \; r_2 = 3\)
\begin{align*}
    a_n = \alpha_12^n + \alpha_23^n
\end{align*}
Frå initialbetingelsane:
\begin{align*}
    \left.\begin{array}{rl}
    a_0 = 1 &= \alpha_1 + \alpha_2 \\
    a_1 = 0 &= \alpha_12 + \alpha_23
    \end{array} \right\} \rightarrow \alpha_1 = 3, \; \alpha_2 = -2
\end{align*}
Som gjev
\begin{align*}
    a_n = 3 \cdot 2^n -2 \cdot 3^n
\end{align*}


\vspace{8mm}
\deloppg{d} \(a_n = 4a_{n-1} - 4a_{n-2}, \quad n \geq 2, \quad a_0 = 6, \; a_1 = 8\)

\noindent Karakteristisk likning: \( r² - 4r + 4 \rightarrow r_0 = 2\)
\begin{align*}
    a_n = \alpha_12^n \alpha_2n2^n
\end{align*}
Frå initialbetingelsane:
\begin{align*}
    \left.\begin{array}{rl}
        a_0 = 6 &= \alpha_1 \\
        a_1 = 8 &= \alpha_12 + \alpha_22
    \end{array} \right\} \rightarrow \alpha_1 = 6, \; \alpha_2 = -2
\end{align*}
Som gjev
\begin{align*}
    a_n &= 6\cdot 2^n - 2 \cdot 2^n \\
    &= 3 \cdot 2^{n+1} - 2^{n + 1}
\end{align*}


\newpage
\deloppg{e} \(a_n = -4a_{n-1} -4a_{n-2}, \quad n \geq 2, \quad a_0 = 0,\; a_1 = 1\)


\noindent Karakteristisk likning: \(r^2 + 4r + 4 \rightarrow r_0 = -2\)
\begin{align*}
    a_n = \alpha_1(-2)^n + \alpha_2n(-2)^n
\end{align*}
Frå initialbetingelsane:
\begin{align*}
    \left.\begin{array}{rl}
        a_0 = 0 &= \alpha_1 \\
        a_1 = 1 &= \alpha_1(-2) + \alpha_2(-2) = \alpha_2(-2)
    \end{array} \right\} \rightarrow \alpha_1 = 0,\; \alpha_2 = -1
\end{align*}
Som gjev:
\begin{align*}
    a_n &= -1 \cdot n(-2)^n
\end{align*}


\vspace{8mm}
\deloppg{g} \(a_n = a_{n-2}/4,\quad n \geq 2,\quad a_0 = 1, \; a_1 = 0\)

\noindent Merk at \(a_n = 0a_{n-1} + \frac{1}{4}a_{n-2}\)

\noindent Karakteristisk likning: \(r^2 - 0r - \frac{1}{4} \rightarrow r_1 = \frac{1}{2}, \; f_2 = -\frac{1}{2}\)
\begin{align*}
    a_n = \alpha_1\left(\frac{1}{2}\right)^n + \alpha_2\left(-\frac{1}{2}\right)^n
\end{align*}
Frå initialbetingelsane:
\begin{align*}
    \left.\begin{array}{rl}
        a_0 = 1 &= \alpha_1 + \alpha_2 \\
        a_1 = 0 &= \frac{\alpha_1}{2} - \frac{\alpha_2}{2}
    \end{array}\right\} \rightarrow \alpha_1 = \alpha_2 = \frac{1}{2}
\end{align*}
Som gjev:
\begin{align*}
    a_n = \left(\frac{1}{2}\right)^{n+1} + (-1)^n\left(\frac{1}{2}\right)^{n+1}
\end{align*}

\newpage
\subsection*{Oppgåve 6}
Skal finne kor mange forskjellige beskjedar kan sendast på \(n\) mikrosekund, samansatt av tre signal --
eitt som er 1 mikrosekund langt, og to som er to mikrosekund lange.

Grunntilfella er \(a_1 = 1,\; a_2 = 3\). På eit mikrosekund kan vi berre sende signalet som er eit mikrosekund langt.
På to mikrosekund kan vi sende to av det første, eller eitt av dei to signala på 2 mikrosekund.

På \(n\) mikrosekund kan vi gjere følgjande: Starte med signalet som er eit mikrosekund langt, og velje
mellom dei \(n - 1\) gjenverande beskjedane, eller starte med eitt av dei to signala som er 2 mikrosekund
lange, og velje mellom dei \(n - 2\) gjenverande beskjedane. På \(n\) mikrosekund har vi altso
\begin{align*}
    a_n = a_{n-1} + 2a_{n-2}
\end{align*}
moglege beskjedar.

\subsection*{Oppgåve 11}
Gjeve
\begin{align*}
    L_n = L_{n-1} + L_{n+2}
\end{align*}
Med initialbetingelsane \(L_0 = 2,\; L_1 = 1\)

\deloppg{a} Skal vise at \(L_n = f_{n-1} + f_{n+1}\) (\(f\) er fibonacci-tal). Brukar induksjon.
\begin{align*}
    \intertext{\bf Induktivt steg}
    \intertext{Antar at for ein \(k\) held følgjande for \(n = k\)}
    L_n &= f_{n-1} + f_{n+1} \\
    L_{n+1} &= f_n + f_{n+2} \\
    \intertext{Dette medfører at for alle \(n > k\) held følgjande}
    L_{n+2} &\ceq{p. def} L_{n+1} + L_n \\
    &= f_{n-1} + f_{n+1} + f_n + f_{n+2} \\
    &= (f_{n-1} + f_n) + (f_{n+1} + f_{n+2}) \\
    &\ceq{p. def} f_{n+1} + f_{n + 3}
\end{align*}
\begin{align*}
    \intertext{\bf Grunntilfelle}
    \intertext{Sjekkar for n = 2}
    L_2 = L_1 + L_0 = 2 + 1 = f_3 + f_1 \\
    L_3 = L_2 + L_1 = 3 + 1 = f_4 + f_2
\end{align*}
Vi ser at påstanden held for alle \(n \geq 2\)


\subsection*{Oppgåve 42}
Skal vise at dersom \(a_n = a_{n-1} + a_{n-2}\), med initialbetingelsane \(a_0 = s, \; a_1 = t\)  så er
\(a_n = sf_{n-1} + sfn\). Induksjkonsbevis.
\begin{align*}
    \intertext{\bf Induktiv del}
    \intertext{Antar at får ein \(k\) held følgande for \(n = k\)}
    a_n &= sf_{n-1} + tf_n \\
    a_{n+1} &= sf_n + tf_{n+1} \\
    \intertext{Dette medfører at for alle \(n > k\) held}
    a_{n+2} &\ceq{p. def} sf_{n-1} + sf_n + tf_n + tf_{n+1} \\
    &= s(f_{n-1} + f_n) + t(f_n + f_{n+1}) \\
    &= sf_{n+1} \ t_{n+2}
\end{align*}
\begin{align*}
    \intertext{Grunntilfelle}
    a_2 &= s + t = sf_1 + sf_2 \\
    a_3 &= s + 2t = sf_2 + sf_3\\
\end{align*}
Vi ser at påstanden held for alle \(n > 2\)


\end{document}
