\documentclass[a4paper, 11pt]{article}
\usepackage{fancyhdr}
\usepackage[all]{nowidow}
\usepackage{graphicx}
\usepackage[dvipsnames]{xcolor}
\usepackage{fontspec}
\usepackage{filecontents}

\setmonofont{Consolas}
\newfontfamily{\Consolas}{Consolas}
\newfontfamily{\Inconsolata}{Inconsolata}
\setmainfont{Linux Libertine}

% Page geometry and layout-related
\usepackage{sectsty}
\usepackage{geometry}
\usepackage{marginnote}
\usepackage[11pt]{moresize}

% Mathematics
\usepackage{amsmath}
\usepackage{amssymb}
\usepackage{array}
\usepackage{mathtools}
\usepackage{bm}
\usepackage{blkarray}

% Code
\usepackage{listings}

% PGF/TikZ
\usepackage{tikz}
\usetikzlibrary{decorations.markings}
\usetikzlibrary{calc,arrows,patterns}
\usepackage{pgfplots}
\usepackage{caption}
\usepackage{subcaption}


% Theme related
\newcommand{\Themecolor}{Black} % main color in theme
\newcommand{\Themetext}[1]{\textcolor{\Themecolor}{#1}}
\newcommand{\TT}[1]{\Themetext{#1}}
\renewcommand{\arraystretch}{1.2}
\definecolor{DarkGray}{rgb}{0.2,0.2,0.2}
\definecolor{MedGray}{rgb}{0.3,0.3,0.3}
\definecolor{LightGray}{rgb}{0.6,0.6,0.6}

% -------------------- %
% Custom functionality %
% -------------------- %

% Document organization
\newcommand{\Task}[1]{
  \noindent {{\bf \Themetext{#1:}}}
}
\newcommand{\MajorTask}[1]{
  \vspace{5mm}
  \reversemarginpar
  \marginnote{\LARGE\bfseries \Themetext{#1}}
}

% Math
\newcommand{\Ceq}[2]{\stackrel{\text{\tiny{#1}}}{#2}} % use for commented =, \leq \geq etc.
\newcommand{\R}{\mathbb{R}}
\newcommand{\Z}{\mathbb{Z}}
\newcommand{\Mod}{\;\text{\bf mod}\;}
\newcommand{\Modulo}[1]{\;(\text{mod }#1)}
\newcommand{\Sgn}[1]{\text{sgn}(#1)}
\newcommand{\Dul}[1]{\underline{\underline{#1}}}


% Få fancyhdr til å holde kjeft
\setlength{\headheight}{14pt} 

% Få pgfplots til å holde kjeft
\pgfplotsset{compat=1.15}

\lstdefinelanguage{Lie}
{
  % list of keywords
  morekeywords={
    include,
    if,
    let,
    fn,
    lambda,
    match,
    unless,
    def,
  },
  sensitive=true,       % keywords are case-sensitive
  morecomment=[l]{;;},  % l is for line comment
  morestring=[b]"       % double quoted strings
}

\lstdefinestyle{themecode}{
    keywordstyle=\Inconsolata\bfseries\Themetext,
    commentstyle=\textcolor{LightGray},
    stringstyle=\textcolor{MedGray}
}

\lstset{
  numbers=left,
  numberstyle=\footnotesize\ttfamily\color{MedGray},
  basicstyle=\ttfamily,
  style=themecode,
  extendedchars=true,
  showstringspaces=false,
  literate=
  {λ}{{\Consolas\TT{λ}}}1
  {Λ}{{\Consolas\TT{Λ}}}1
  {σ}{{\Consolas {σ}}}1
  {Σ}{{\Consolas {Σ}}}1
  {<}{{\Consolas {<}}}1
  {>}{{\Consolas {>}}}1
  {-}{{\Consolas {-}}}1
  {+}{{\Consolas {+}}}1
  {π}{{\Consolas\textcolor{DarkGray}{π}}}1
  {Π}{{\Consolas\textcolor{DarkGray}{Π}}}1
  {True}{{\textbf{\textcolor{DarkGray}{True}}}}4
  {False}{{\textbf{\textcolor{DarkGray}{False}}}}5
  {NIL}{{\textbf{\textcolor{DarkGray}{NIL}}}}3,
}


% This is (by far btw) the worlds ugliest hack
% Makes some unicode work in lstlistings
% TODO: put this shit in a file
\makeatletter
\lst@InputCatcodes
\def\lst@DefEC{%
  \lst@CCECUse \lst@ProcessLetter
  ^^^^03bb^^^^039b % lambda
  ^^^^03c0^^^^03a0 % pi
  ^^^^03c3^^^^03a3 % sigma
  ^^^^03b1^^^^0391 % alpha
  ^^^^03b2^^^^0392 % beta
  ^^00}
\lst@RestoreCatcodes
\makeatother

\begin{document}
% Front Page:
{
  \raggedleft
  {\Large Steffen Haug} \\
  {\large \underline{\bf Rettast}} \\ [0.167\textheight]
  {\HUGE \bfseries \Themetext{Øving 9}} \\ [\baselineskip]
  {\LARGE {Diskret Matematikk}} \\
}
\thispagestyle{empty}
\pagestyle{empty}
\newpage

\section*{Oppgåver til seksjon 9.6} \vspace{-5mm}
\MajorTask{9} \noindent Skal avgjera om relasjonen med den følgjande retta grafen
er ei partiell ordning.
Skal med andre ord avgjera om relasjonen er refleksiv, antisymmetrisk og transitiv.

\begin{figure}[h]
  \centering
  \begin{tikzpicture}[scale=2]
    \tikzset{
      ->-/.default=.615,
      ->-/.style={decoration={
          markings,
          mark=at position #1 with {\arrow{>}}},
        postaction={decorate}
      }
    }
    \tikzset{edge/.style={->-=#1,thick,shorten <= 2pt,shorten >= 2pt}}
    \coordinate (A) at (0,0);
    \coordinate (B) at (1,0);
    \coordinate (C) at (0,-1);
    \coordinate (D) at (1,-1);
    \draw[edge=.5] (A) -- (B);
    \draw[edge] (A) -- (C);
    \draw[edge] (B) -- (D);
    \draw[edge=.5] (A) arc [start angle=-45, delta angle=360, radius=5pt];
    \draw[edge=.5] (B) arc [start angle=225, delta angle=-360, radius=5pt];
    \draw[edge=.5] (C) arc [start angle=45, delta angle=360, radius=5pt];
    \draw[edge=.5] (D) arc [start angle=135, delta angle=-360, radius=5pt];
    \draw[fill=white] (A) circle (1pt) node [below right] {\(a\)};
    \draw[fill=white] (B) circle (1pt) node [below left] {\(b\)};
    \draw[fill=white] (C) circle (1pt) node [above right] {\(c\)};
    \draw[fill=white] (D) circle (1pt) node [above left] {\(d\)};
  \end{tikzpicture}
\end{figure}

\noindent \textbf{Refleksivitet:} Alle nodar i grafen har kantar til seg sjølv.
Dette betyr at \((x,x) \in R\) for alle \(x \in S\). Relasjonen er refleksiv.

\noindent \textbf{Antisymmetri:} Relasjonen har ingen distinkte par \(x,y\) slik
at både \((x,y) \in R\) og \((y,x) \in R\).
Med andre ord: \((x,y) \in R \implies (y,x) \not\in R\). Relasjonen er antisymmetrisk.

\noindent \textbf{Transitivitet:} Relasjonen inneheld \((a, b)\) og \((b,d)\), men ikkje
\((a,d)\).
Det er altso ikkje slik at \((x,y) \in R, (y,z) \in R \implies (x,z) \in R\).
Relasjonen er {\em ikkje} transitiv.

\vspace{3mm}\noindent
Sidan relasjonen ikkje er transitiv på \(S\), er den ikkje ei partiell ordning av \(S\).


\MajorTask{18}
\Task{b} Skal finne den alfabetiske ordninga av orda
\begin{align*}
  \text{open, opener, opera, operand, opened}
\end{align*}
Som er
\begin{align*}
  \text{\Themetext{open, opened, opener, opera, operand}}
\end{align*}


\MajorTask{27} \noindent
Skal finne alle par i den partielle ordninga med følgjande Hasse-diagram.

\vspace{2mm}
\begin{minipage}{.5\textwidth}
  \centering
  \begin{tikzpicture}[scale=2]
    \tikzset{edge/.style = {thick,shorten <=2pt,shorten >=2pt}}
    \coordinate (D) at (0,0);
    \coordinate (E) at (1,0);
    \coordinate (F) at (2,0);
    \coordinate (G) at (1,-.75);
    \coordinate (B) at (0,-1.5);
    \coordinate (A) at (1,-1.5);
    \coordinate (C) at (2,-1.5);

    \draw[edge] (A) -- (G);
    \draw[edge] (B) -- (G);
    \draw[edge] (C) -- (G);
    \draw[edge] (G) -- (D);
    \draw[edge] (G) -- (E);
    \draw[edge] (G) -- (F);
    
    \draw[fill=white] (A) circle (1pt) node [below] {\(a\)};
    \draw[fill=white] (B) circle (1pt) node [below] {\(b\)};
    \draw[fill=white] (C) circle (1pt) node [below] {\(c\)};
    \draw[fill=white] (D) circle (1pt) node [above] {\(d\)};
    \draw[fill=white] (E) circle (1pt) node [above] {\(e\)};
    \draw[fill=white] (F) circle (1pt) node [above] {\(f\)};
    \draw[fill=white] (G) circle (1pt) node [right] {\(\;g\)};
  \end{tikzpicture}
\end{minipage}
\begin{minipage}{.5\textwidth}
\begin{align*}
  R = \{&(b,g), (b,d), (b,e), (b,f), \\
        &(a,g), (a,d), (a,e), (a,f), \\
        &(c,g), (c,d), (c,e), (c,f)\}
\end{align*}
\end{minipage}


\newpage
\MajorTask{32} \noindent
Gjeve Hasse-diagrammet

\vspace{2mm}
\begin{minipage}{.5\textwidth}
  \centering
  \begin{tikzpicture}[scale=2]
    \tikzset{edge/.style = {thick,shorten <=2pt,shorten >=2pt}}
    \coordinate (A) at (0,0);
    \coordinate (B) at (1,0);
    \coordinate (C) at (2,0);
    \coordinate (D) at (0,1);
    \coordinate (E) at (1,1);
    \coordinate (F) at (2,1);
    \coordinate (G) at (2,2);
    \coordinate (H) at (1,2);
    \coordinate (I) at (0,2);
    \coordinate (J) at (0,3);
    \coordinate (K) at (1,3);
    \coordinate (L) at (0,4);
    \coordinate (M) at (2,4);

    \draw[edge] (A) -- (D);
    \draw[edge] (B) -- (D);
    \draw[edge] (B) -- (E);
    \draw[edge] (C) -- (F);

    \draw[edge] (D) -- (I);
    \draw[edge] (D) -- (H);
    \draw[edge] (E) -- (H);
    \draw[edge] (F) -- (G);

    \draw[edge] (I) -- (J) -- (L);
    \draw[edge] (H) -- (K) -- (L);
    \draw[edge] (H) -- (J);
    \draw[edge] (G) -- (K) -- (M);
    
    \draw[fill=white] (A) circle (1pt) node [left] {\(a\)};
    \draw[fill=white] (B) circle (1pt) node [right] {\(b\)};
    \draw[fill=white] (C) circle (1pt) node [right] {\(c\)};
    \draw[fill=white] (D) circle (1pt) node [left] {\(d\)};
    \draw[fill=white] (E) circle (1pt) node [right] {\(e\)};
    \draw[fill=white] (F) circle (1pt) node [right] {\(f\)};
    \draw[fill=white] (G) circle (1pt) node [right] {\(g\)};
    \draw[fill=white] (H) circle (1pt) node [right] {\(h\)};
    \draw[fill=white] (I) circle (1pt) node [left] {\(i\)};
    \draw[fill=white] (J) circle (1pt) node [left] {\(j\)};
    \draw[fill=white] (K) circle (1pt) node [right] {\(k\)};
    \draw[fill=white] (L) circle (1pt) node [left] {\(l\)};
    \draw[fill=white] (M) circle (1pt) node [right] {\(m\)};
  \end{tikzpicture}
\end{minipage}
\begin{minipage}{.45\textwidth}
  \Task{a} Kva er dei største elementa?

  \(l\) og \(m\)

  \vspace{2mm}
  \Task{b} Kva er dei minste elementa?

  \(a\), \(b\) og \(c\)

  \vspace{2mm}
  \Task{c} Er der eit største element?

  Nei, ordninga seier ingenting om kva som er størst av \(l\) og \(m\),
  dermed har den ikkje eit unikt største element.

  \vspace{2mm}
  \Task{d} Er der eit minste element?

  Nei, ordninga seier ingenting om kva som er minst av \(a\), \(b\) og \(c\),
  dermed har den ikkje eit unikt minste element.

  \vspace{2mm}
  \Task{e} Finn alle øvre grenser av \(\{a,b,c\}\)

  \(\{k,l,m\}\) er øvre skranker av \(\{a,b,c\}\)

\end{minipage}

\Task{f} Finn den minste øvre grensa av \(\{a,b,c\}\) (dersom den eksisterer)

\noindent
\(k\) er øvre grense, og er unikt minst av desse, altso er \(k\) minste
øvre grense.

\vspace{2mm}
\Task{g} Finn alle nedre grenser av \(\{f,g,h\}\)

\noindent
\(h\) har ingen felles nedre grense med \(f\) og \(g\), dermed har \(\{f,g,h\}\)
nedre grenser \(\emptyset\)

\vspace{2mm}
\Task{h} Finn største nedre grense av \(\{f,g,h\}\) (dersom den eksisterer)

\noindent
Eksisterer ikkje


\section*{Oppgåver til seksjon 10.2} \vspace{-5mm}
\MajorTask{18} \noindent
Skal vise at i ein enkel graf må minst to nodar ha same grad.

\vspace{3mm}\noindent
I ein enkel graf er det ikkje tillete at ei node kan ha meir enn ein kant
til ei anna node. I ein enkel graf er det heller ingen løkker.
I ein enkel graf på \(n\) nodar kan ei node derfor, på det meste, ha kantar
berre til dei andre \(n-1\) nodene.

\vspace{1mm} \noindent
Sidan grafen har \(n\) nodar, og kvar av dei maksimalt kan ha \(n-1\) kantar,
har vi av fugleburprinsippet at {\em minst to} av nodane må ha like mange kantar,
altso har dei same grad.


\newpage
\MajorTask{22} \noindent
Skal avgjere om grafen er todelt (er det dette det heiter på norsk?).
Bruker Teorem 4, og fargar nodane i grafen.
\begin{figure}[h]
  \centering
  \begin{tikzpicture}
    \tikzset{edge/.style = {thick,shorten <= 2pt,shorten >= 2pt,>=stealth}}
    \coordinate (a) at (0,1);
    \coordinate (b) at (1,2);
    \coordinate (c) at (3,2);
    \coordinate (d) at (4,1);
    \coordinate (e) at (2,0);

    \draw[edge] (a) -- (d);
    \draw[edge] (a) -- (b) -- (c) -- (d);
    \draw[edge] (a) -- (e) -- (c);

    \draw[fill=black] (a) circle (2pt) node [left] {$a$};
    \draw[fill=white] (b) circle (2pt) node [above] {$b$};
    \draw[fill=black] (c) circle (2pt) node [above] {$c$};
    \draw[fill=white] (d) circle (2pt) node [right] {$d$};
    \draw[fill=white] (e) circle (2pt) node [below] {$e$};
  \end{tikzpicture}
\end{figure}

\noindent
Vi ser at grafen er todelt.


\MajorTask{26}
\Task{a} \(K_n\) er berre todelt for \(n=1,2\).
\(K_3\) er ikkje todelt (sjå figur; vi kan ikkje farge den grå noden),
og \(K_3\) er delgraf i alle \(K_n,\; n \geq 3\).
Dermed er ingen andre komplette grafar todelte.
\begin{figure}[h]
  \centering
  \begin{tikzpicture} 
    \coordinate (a) at (0,0);
    \coordinate (b) at (2,0);
    \coordinate (c) at (1,1.732);

    \draw (a) -- (b) -- (c) -- cycle;
    
    \draw[fill=black] (a) circle (2pt);
    \draw[fill=white] (b) circle (2pt);
    \draw[fill=gray] (c) circle (2pt);
  \end{tikzpicture}
\end{figure}

\vspace{3mm}
\Task{b} \(C_n\) er todelt for partal \(n\). Dette ser vi enkelt dersom
vi teiknar sykliske grafar på ein spesiell måte:

\begin{center}
\begin{minipage}{.4\textwidth}
  \centering
  \begin{tikzpicture}[scale=.5]
    \coordinate (a) at (0,0);
    \coordinate (b) at (0,2);
    \coordinate (c) at (2,0);
    \coordinate (d) at (2,2);
    \coordinate (e) at (4,0);
    \coordinate (f) at (4,2);
    \coordinate (g) at (6,0);
    \coordinate (h) at (6,2);
    \coordinate (i) at (8,0);
    \coordinate (j) at (8,2);

    \draw (a)--(b)--(d)--(f);
    \draw[dotted] (f)--(h);
    \draw (h)--(j)--(i)--(g);
    \draw[dotted] (g)--(e);
    \draw (e)--(c)--(a);

    \draw[->,shorten <= 6pt, shorten >= 6pt] (d) to [bend left] (c);
    \draw[->,shorten <= 6pt, shorten >= 6pt] (c) to [bend left] (d);
    \draw[->,shorten <= 6pt, shorten >= 6pt] (h) to [bend left] (g);
    \draw[->,shorten <= 6pt, shorten >= 6pt] (g) to [bend left] (h);

    \draw[fill=black] (a) circle (4pt);
    \draw[fill=white] (b) circle (4pt);
    \draw[fill=white] (c) circle (4pt);
    \draw[fill=black] (d) circle (4pt);
    \draw[fill=black] (e) circle (4pt);
    \draw[fill=white] (f) circle (4pt);
    \draw[fill=white] (g) circle (4pt);
    \draw[fill=black] (h) circle (4pt);
    \draw[fill=black] (i) circle (4pt);
    \draw[fill=white] (j) circle (4pt);
  \end{tikzpicture}
\end{minipage}
\(\rightarrow\)
\begin{minipage}{.4\textwidth}
  \centering
  \begin{tikzpicture}[scale=.5]
    \coordinate (a) at (0,0);
    \coordinate (b) at (0,2);
    \coordinate (c) at (2,2);
    \coordinate (d) at (2,0);
    \coordinate (e) at (4,0);
    \coordinate (f) at (4,2);
    \coordinate (g) at (6,2);
    \coordinate (h) at (6,0);
    \coordinate (i) at (8,0);
    \coordinate (j) at (8,2);

    \draw[gray, dashed] (-1,1) -- (9,1);

    \draw (a)--(b)--(d)--(f);
    \draw[dotted] (f)--(h);
    \draw (h)--(j)--(i)--(g);
    \draw[dotted] (g)--(e);
    \draw (e)--(c)--(a);

    \draw[fill=black] (a) circle (4pt);
    \draw[fill=white] (b) circle (4pt);
    \draw[fill=white] (c) circle (4pt);
    \draw[fill=black] (d) circle (4pt);
    \draw[fill=black] (e) circle (4pt);
    \draw[fill=white] (f) circle (4pt);
    \draw[fill=white] (g) circle (4pt);
    \draw[fill=black] (h) circle (4pt);
    \draw[fill=black] (i) circle (4pt);
    \draw[fill=white] (j) circle (4pt);
  \end{tikzpicture}
\end{minipage}
\end{center}

\noindent Sykliske grafar med odde \(n\) er ikkje todelte:
\begin{figure}[h]
  \centering
  \begin{tikzpicture}[scale=.5]
    \coordinate (a) at (0,0);
    \coordinate (b) at (0,2);
    \coordinate (c) at (2,2);
    \coordinate (d) at (2,0);
    \coordinate (e) at (4,0);
    \coordinate (f) at (4,2);
    \coordinate (g) at (6,2);
    \coordinate (h) at (6,0);
    \coordinate (i) at (8,0);
    \coordinate (j) at (8,2);

    \coordinate (k) at (9.73, 1);

    \draw (a)--(b)--(d)--(f);
    \draw[dotted] (f)--(h);
    \draw (h)--(j);
    \draw[dashed] (j)--(i);
    \draw (i)--(g);
    \draw[dotted] (g)--(e);
    \draw (e)--(c)--(a);
    \draw (i)--(k)--(j);

    \draw[fill=black] (a) circle (4pt);
    \draw[fill=white] (b) circle (4pt);
    \draw[fill=white] (c) circle (4pt);
    \draw[fill=black] (d) circle (4pt);
    \draw[fill=black] (e) circle (4pt);
    \draw[fill=white] (f) circle (4pt);
    \draw[fill=white] (g) circle (4pt);
    \draw[fill=black] (h) circle (4pt);
    \draw[fill=black] (i) circle (4pt);
    \draw[fill=white] (j) circle (4pt);
    \draw[fill=gray] (k) circle (4pt);
  \end{tikzpicture}
\end{figure}

\noindent
Uavhengig av kvar i grafen vi set inn ei ny node må vi plassere den mellom
ei svart og ei kvit, og dermed kan den ikkje fargast verken svart eller kvit.

\vspace{3mm}
\Task{c} \(W_n\) er aldri todelt. \(W_n\) har alltid \(C_n\) som undergraf med \(n \geq 3\),
og sjølv når denne er todelt kan ein ikkje legge til ei node med kantar
til alle nodene i \(C_n\) slik at det er mogleg å farge den svart eller kvit
j.f. Teorem 4.


\newpage
\MajorTask{55} \noindent
Skal finne kor mange nodar ein regulær graf med 10 kantar og grad 4 har
\begin{figure}[h]
  \centering
  \begin{tikzpicture}
    \coordinate (a1) at (270:2);
    \coordinate (a2) at (342:2);
    \coordinate (a3) at (54:2);
    \coordinate (a4) at (126:2);
    \coordinate (a5) at (198:2);

    \draw[thick] (a1)
    to [bend right] (a2)
    to [bend right] (a3)
    to [bend right] (a4)
    to [bend right] (a5)
    to [bend right] cycle;
    \draw[thick] (a1) --(a3) -- (a5) -- (a2) -- (a4) -- cycle;
    
    \draw[fill=white] (a1) circle (2pt);
    \draw[fill=white] (a2) circle (2pt);
    \draw[fill=white] (a3) circle (2pt);
    \draw[fill=white] (a4) circle (2pt);
    \draw[fill=white] (a5) circle (2pt);
  \end{tikzpicture}
\end{figure}

\noindent
Når grafen er so enkel klarer vi å løyse problemet utan å vera spesielt kreativ i framgangsmåten.
Antal kantar i komplette grafar med \(n\) nodar er
\begin{align*}
  E(K_n) = \sum_{i=0}^{n-1}i=\sum_{i=1}^{n}(i - 1) = \frac{n^2 -n}{2}
\end{align*}
Fordi den \(n\)-te noden har kantar til dei \(n-1\) andre, den \(n-1\)-te noda
har kantar til \(n-2\) nodar, fordi kanten til den \(n\)-te allereie er tald.
\(K_n\) har grad \(n-1\).

\vspace{2mm} \noindent
Sidan vi kjenner den øvrige summen for små \(n\) er det enkelt å finne grafen.
Større grafar hadde vore vanskeleg å finne på ein like naiv måte.


\section*{Oppgåver til seksjon 10.3} \vspace{-5mm}
\MajorTask{17} \noindent
Skal teikne grafen med nabomatrisa

\begin{center}
\begin{minipage}{.4\textwidth}
  \begin{align*}
    \begin{bmatrix}
      1 & 2 & 0 & 1 \\
      2 & 0 & 3 & 0 \\
      0 & 3 & 1 & 1 \\
      1 & 0 & 1 & 0
    \end{bmatrix}
  \end{align*}
\end{minipage}
\(\rightarrow\)
\begin{minipage}{.4\textwidth}
  \centering
  \begin{tikzpicture}[scale=1]
    \tikzset{edge/.style = {thick,shorten <= 2pt,shorten >= 2pt,>=stealth}}
    \coordinate (a) at (2,0);
    \coordinate (b) at (2,-2);
    \coordinate (c) at (4,-2);
    \coordinate (d) at (4,0);

    \draw[edge] (a) arc [start angle=-45, delta angle=360, radius=.5];
    \draw[edge] (a) to [bend right] (b);
    \draw[edge] (a) to [bend left] (b);
    \draw[edge] (b) to (c);
    \draw[edge] (b) to [bend left] (c);
    \draw[edge] (b) to [bend right] (c);

    \draw[edge] (c) arc [start angle=135, delta angle=360, radius=.5];

    \draw[edge] (c) to (d);
    \draw[edge] (a) to (d);

    \draw[fill=white] (a) circle (2pt) node [above left] {$a$};
    \draw[fill=white] (b) circle (2pt) node [below left] {$b$};
    \draw[fill=white] (c) circle (2pt) node [below right] {$c$};
    \draw[fill=white] (d) circle (2pt) node [above right] {$d$};
  \end{tikzpicture}
\end{minipage}
\end{center}


\newpage
\MajorTask{19} \noindent
Skal finne nabomatrisa til grafen

\begin{center}
\begin{minipage}{.4\textwidth}
  \centering
  \begin{tikzpicture}[scale=1]
    \tikzset{
      ->-/.default=.615,
      ->-/.style={decoration={
          markings,
          mark=at position #1 with {\arrow{>}}},
        postaction={decorate}
      }
    }
    \tikzset{edge/.style={->-=#1,thick}}
    \coordinate (a) at (0,0);
    \coordinate (b) at (2,0);
    \coordinate (c) at (0,-2);
    \coordinate (d) at (2,-2);


    \draw[edge] (a) to (b);

    \draw[edge] (b) to [bend right] (c);
    \draw[edge] (c) to [bend right] (b);

    \draw[edge] (d) to (a);
    \draw[edge] (c) to (d);

    \draw[edge=.5] (c) arc [start angle=45, delta angle=360, radius=.5];
    \draw[edge=.5] (b) arc [start angle=-135, delta angle=360, radius=.5];

    \draw[fill=white] (a) circle (2pt) node [above left] {$a$};
    \draw[fill=white] (b) circle (2pt) node [above right] {$b$};
    \draw[fill=white] (c) circle (2pt) node [below left] {$c$};
    \draw[fill=white] (d) circle (2pt) node [below right] {$d$};
  \end{tikzpicture}
\end{minipage}
\(\rightarrow\)
\begin{minipage}{.4\textwidth}
  \begin{align*}
    \begin{bmatrix}
      0 & 1 & 0 & 0 \\
      0 & 1 & 1 & 0 \\
      0 & 1 & 1 & 1 \\
      1 & 0 & 0 & 0
    \end{bmatrix}
  \end{align*}
\end{minipage}
\end{center}


\MajorTask{23} \noindent
Skal teikne grafen med nabomatrisa
\begin{center}
\begin{minipage}{.4\textwidth}
  \begin{align*}
    \begin{bmatrix}
      1 & 2 & 1 \\
      2 & 0 & 0 \\
      0 & 2 & 2
    \end{bmatrix}
  \end{align*}
\end{minipage}
\(\rightarrow\)
\begin{minipage}{.4\textwidth}
  \centering
  \begin{tikzpicture}[scale=1]
    \tikzset{
      ->-/.default=.615,
      ->-/.style={decoration={
          markings,
          mark=at position #1 with {\arrow{>}}},
        postaction={decorate}
      }
    }
    \tikzset{edge/.style={->-=#1,thick,shorten <= 2pt,shorten >= 2pt}}
    \coordinate (a) at (0,0);
    \coordinate (b) at (4,0);
    \coordinate (c) at (2,3.46);

    \draw[edge=.5] (a) arc [start angle=45, delta angle=360, radius=.4];

    \draw[edge] (a) to [bend left=12] (b);
    \draw[edge] (a) to [bend left] (b);

    \draw[edge] (b) to [bend left] (a);
    \draw[edge] (b) to [bend left=12] (a);

    \draw[edge] (c)  to [bend left] (b);
    \draw[edge] (c)  to [bend right] (b);
  
    \draw[edge] (a) to (c);
    
    \draw[edge=.5] (c) arc [start angle=-90, delta angle=360, radius=.6];
    \draw[edge=.5] (c) arc [start angle=-90, delta angle=360, radius=.4];

    \draw[fill=white] (a) circle (2pt) node [below left] {$a$};
    \draw[fill=white] (b) circle (2pt) node [below right] {$b$};
    \draw[fill=white] (c) circle (2pt) node [above] {$c$};
  \end{tikzpicture}
\end{minipage}
\end{center}

\end{document}