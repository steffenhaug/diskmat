\documentclass[a4paper, 11pt]{article}
\usepackage[T1]{fontenc}
\usepackage[utf8]{inputenc}
\usepackage[norsk]{babel}
\usepackage{graphicx}
\usepackage[dvipsnames]{xcolor}
\usepackage{amsmath}
\usepackage{amsthm}
\usepackage{amssymb}
\usepackage{microtype}
\usepackage{array}
\usepackage{caption}
\usepackage{subcaption}
\usepackage[norsk]{babel}
\usepackage[all]{nowidow}
\usepackage{fancyhdr}
\usepackage{mathtools}
\usepackage{sectsty}
\usepackage{etoolbox}
\usepackage{cancel}
\usepackage{listings}
\usepackage{inconsolata}
\usepackage{tikz}
\usetikzlibrary{calc,arrows,patterns}
\usepackage{pgfplots}


% Unicode-inkompatibilitet mactex/texlive?
\DeclareUnicodeCharacter{00A0}{ }

% Få fancyhdr til å holde kjeft
\setlength{\headheight}{14pt} 

\newcommand{\tittel}{Øving 5}
\newcommand{\fag}{Diskret Matematikk}
\newcommand{\fagkode}{TMA4140}
\newcommand{\forfatter}{Steffen Haug}

\renewcommand{\qedsymbol}{$\themecolor{\blacksquare}$}

\newcommand{\ceq}[2]{\stackrel{\text{\tiny{#1}}}{#2}}
\newcommand{\R}{\mathbb{R}}
\newcommand{\Z}{\mathbb{Z}}
\newcommand{\Mod}{\;\text{\bf mod}\;}
\newcommand{\Modulo}{\text{mod }}
\newcommand{\deloppg}[1]{\vspace{1mm}\noindent \textbf{\tt \themecolor{#1:}}}
\newcommand{\dint}{\int\displaylimits}

\newcommand{\themeshade}{Mahogany}
\newcommand{\themecolor}[1]{\textcolor{\themeshade}{#1}}
\sectionfont{\color{\themeshade}}

\def\dul#1{\underline{\underline{#1}}}

% Forside
\newcommand*{\titleTH}{\begingroup 
    \raggedleft
    \vspace*{\baselineskip}
    {\Large \forfatter}\\[0.167\textheight]
    {\LARGE\bfseries \tittel}\\[\baselineskip]
    {\themecolor{\Huge \fag}}\\[\baselineskip]
    {\Large \textit{\fagkode}}\par
    \vfill
\endgroup}

\begin{document}

\lstdefinestyle{themecode}{
    keywordstyle=\themecolor,
}

\lstset{numbers=left, basicstyle=\ttfamily, style=themecode}

% Front Page
\pagestyle{empty}
\titleTH
\newpage


\section*{Oppgåver til seksjon 4.4} % 21 33 37a

\subsection*{Oppgåve 21}
Skal finne løysing til systemet
\begin{align*}
    x &\equiv 1 \;(\Modulo 2) \\
    x &\equiv 2 \;(\Modulo 3) \\
    x &\equiv 3 \;(\Modulo 5) \\
    x &\equiv 4 \;(\Modulo 11)
\end{align*}
La \(m = 2 \cdot 3 \cdot 5 \cdot 11 = 330, M_1 = m/2 = 165, M_2 = m/3 = 110, M_3 = m/5 = 66, M_4 = m/11 = 30\)
Vi ser at
\begin{align*}
    1 \text{ er invers til } &M_1 = 165 \;(\Modulo 2) \text{ fordi } 165 \cdot 1 \equiv 1 \cdot 1 \equiv 1 \;(\Modulo 2) \\
    2 \text{ er invers til } &M_2 = 110 \;(\Modulo 3) \text{ fordi } 110 \cdot 2 \equiv 2 \cdot 2 \equiv 1 \;(\Modulo 3) \\
    1 \text{ er invers til } &M_3 = 66 \;(\Modulo 5) \text{ fordi } 66 \cdot 1 \equiv 1 \cdot 1 \equiv 1 \;(\Modulo 5) \\
    7 \text{ er invers til } &M_4 = 30 \;(\Modulo 11) \text{ fordi } 30 \cdot 7 \equiv 8 \cdot 7 \equiv 1 \;(\Modulo 11)
\end{align*}
Løysingar til systemet er dei slik at
\begin{align*}
    x \equiv a_1M_1y_1 + a_2M_2y_2 + a_3M_3y_3 + a_4M_4y_4
        &= 1 \cdot 165 \cdot 2 + 2 \cdot 110 \cdot 2 + 3 \cdot 66 \cdot 1 + 4 \cdot 30 \cdot 7 \\
        &= 1808 = 158 \;(\Modulo 330)
\end{align*}

\section*{Oppgåve 33}
Skal bruke Fermats vesle teorem (heretter f.v.t.) til å finne \(7^{121} \Mod 13\).
f.v.t. seier at \(7^{12} \equiv 1 \;(\Modulo 13) \rightarrow (7^{12})^n \equiv 1 \;(\Modulo 13)\)
For å nytte dette delar vi eksponenten på 12, og merkar at \(121 = 12 \cdot 10 + 1\). Vi har at
\begin{align*}
    7^{121} = 7^{12 \cdot 10 + 1} = (7^{12})^{10}7 \equiv 1 \cdot 7 \equiv 7 \;(\Modulo 13)
\end{align*}
Dermed \(7^{121} \Mod 13 = 7\)


\subsection*{37a}
Skal vise at \(2^{340} \equiv 1 \;(\Modulo 11)\).
f.v.t gir direkte \((2^{10})^n \equiv 1 \;(\Modulo 11)\).
Dermed har vi \(2^{340} = (2^{10})^{34} \equiv 1 \;(\Modulo 11)\)


\newpage
\section*{Oppgåver til seksjon 6.1} % 27 44
\subsection*{Oppgåve 27}
Skal finne kor mange måtar ein kan organisere ein kommitté av representantar frå 50 statar,
der kvar stat kan representerast av guvernøren {\em eller} ein av to senatorar.

Frå kvar stat er der 3 personar som vera representant, uavhenging av andre statar.
Produktregelen gjev at kommitéen kan setjast saman på \(3^{50}\) forskjellige måtar.


\subsection*{Oppgåve 44}
Skal finne kor mange måtar ein kan plassere 4 personar av ei gruppe på 10 ved eit rundt bord,
der to plasseringar er like dersom kvar person har den same personen på begge sider.

Vi vel eit vilkårleg sete som sete 1, og nummerer setene med klokka.
Det er 10 måtar å velje ein person i sete 1, 9 måtar å velje sete 2, og so vidare.
Dersom vi flyttar alle personar med klokka (vi vel med andre ord eit anna sete som sete 1)
er plasseringa for øvrig den same. Talet måtar (\(n\)) vi kan plassere personane er
\begin{align*}
    n = \frac{10 \cdot 9 \cdot 8 \cdot 7}{4} = 10 \cdot 9 \cdot 7 \cdot 2 = 1260
\end{align*}
Fordi vi kan velje sete 1 på 4 forskjellige måtar.


\section*{Oppgåver til seksjon 6.2} % 10 18
\subsection*{Oppgåve 10}
Skal vise at dersom vi har 5 punkt \((x_i, y_i), \; i = 1,2,3,4,5\), må minst eitt midtpunkta mellom
dei ha heiltalskoordinatar, det vil seie
\begin{align*}
    x_i+x_j, \text{ og } y_i+y_j
\end{align*}
Skal begge vera partal for (minst) to punkt, \(i\) og \(j\).

Merk at der kun er fire måtar å konstruere heiltalspunkt, med omsyn til odde- og partal:
(o, p), (p, o), (p, p), og (o, o).
Med fire punkt på nøyaktig denne forma klarar vi å unngå at begge koordinatane summerar
til partal, men med ein gong vi introduserer eit nytt punkt, uansett kva for ei av dei
fire formane det tek, klarer vi ikkje å unngå det.


\newpage
\subsection*{Oppgåve 18}
Gjeve ei klasse med 9 elevar. Skal vise at

\deloppg{a} Klassa har minst 5 jenter {\em eller} minst 5 guttar.

Dersom klassa har maksimalt 4 jenter og fire gutar kan det berre vera 8 elevar i klassa.

\deloppg{b} Klassa har minst 3 gutar {\em eller} minst 7 jenter.

Dersom klassa har maksimalt 2 gutar og 6 jenter kan det ikkje vera meir enn 8 elevar i klassa.


\section*{Oppgåver til seksjon 6.3}
\subsection*{Oppgåve 13}
Skal finne ut kor mange måtar ein kan organisere ein kø av \(n\) menn og \(n\) kvinner, slik at
menn og kvinner alternerer.

Køen består med andre ord av kvinner og menn organisert parvis. Det fremste paret vert trekte frå
\(n\) menn og \(n\) kvinner (\(n^2\) måtar å organisere), det andre paret vert trekte frå \(n-1\)
menn og \(n-1\) kvinner (\((n-1)^2\) måtar å organisere), og so vidare, så heile køen kan organiserast på
\(n^2(n-1)^2(n-2)^2 \cdots 2 \cdot 1\) måtar. Vi kjenner summen av \(n\) kvadrat, og den er
\begin{align*}
    \frac{n(n+1)(2n+1)}{6}
\end{align*}


\subsection*{Oppgåve 19}
\begin{lstlisting}
(fn fac [n]
  (if (zero? n)
    1
    (* n (fac (dec n)))))

(fn ncr [n r]
  (/ (fac n) (* (fac r) (fac (- n r)))))
\end{lstlisting}

\deloppg{b}
Skal finne kor mange utfall som har nøyaktig 2 kron, når ein mynt vert kasta 10 gongar.

Rekkjefølgja vi ``trekk'' myntane i betyr ikkje noko. Svaret er altal 2-kombinasjonar av 10:
\begin{align*}
    C(10,2) = 45
\end{align*}

\deloppg{c}
Skal finne kor mange utfall som har maksimalt 3 mynt, når ein mynt vert kasta 10 gongar.

Rekkjefølgja vi ``trekk'' myntane i betyr ikkje noko. Svaret er summen av kombinasjonane
som gjer nøyaktig 1, 2 og 3 myntar.
\begin{align*}
    C(10,1) + C(10,2) + C(10,3) = 175
\end{align*}


\subsection*{Oppgåve 34}
Gjeve ei avdeling med 10 menn og 15 kvinner, skal finne kor mange måtar ein kan organisere ein
kommitté på 6 personar slik at den har fleire kvinner enn menn. Kommittéen kan med andre ord ha
4 kvinner og 2 menn, 5 kvinner og 1 mann, eller 6 kvinner.
\begin{align*}
    C(15,4) \cdot C(10,2) + C(15,5) \cdot C(10,1) + C(15,6) \cdot C(10, 0) = 96460
\end{align*}


\section*{Oppgåver til seksjon 6.4}
\subsection*{Oppgåve 9}
Skal finne koeffisienten til \(x^{101}y^{99}\) i \((2x - 3y)^{200}\).
\begin{align*}
    (2x + (-3y))^{200} = \sum_{j = 0}^{200}{200 \choose j} (2x)^{200-j}(-3y)^j
\end{align*}
Koeffisienten vi skal finne svarar til \(j = 99\), nemleg
\begin{align*}
    {200 \choose 99}2^{101}(-3)^{99} = -\frac{200!}{101!99!}2^{101}(-3)^{99}
\end{align*}



\end{document}
