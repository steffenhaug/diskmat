\documentclass[a4paper, 11pt]{article}
\usepackage[T1]{fontenc}
\usepackage[utf8]{inputenc}
\usepackage{microtype}
\usepackage{fancyhdr}
\usepackage[all]{nowidow}
\usepackage{graphicx}
\usepackage[dvipsnames]{xcolor}

% Page geometry and layout-related
\usepackage{sectsty}
\usepackage{geometry}
\usepackage{marginnote}
\usepackage[11pt]{moresize}

% Mathematics
\usepackage{amsmath}
\usepackage{amssymb}
\usepackage{array}
\usepackage{mathtools}
\usepackage{bm}
\usepackage{blkarray}

% Code
\usepackage{listings}
\usepackage{inconsolata}

% PGF/TikZ
\usepackage{tikz}
\usetikzlibrary{calc,arrows,patterns}
\usepackage{pgfplots}


% Theme related
\newcommand{\Themecolor}{Mahogany} % main color in theme
\newcommand{\Themetext}[1]{\textcolor{\Themecolor}{#1}}
\newcommand{\TT}[1]{\Themetext{#1}}
\renewcommand{\arraystretch}{1.2}


% Custom functionality
% --------------------

% Document organization
\newcommand{\Task}[1]{\vspace{3mm}\noindent {\tt \Themetext{#1:}}}
\newcommand{\MajorTask}[1]{
  \vspace{5mm}
  \reversemarginpar
  \marginnote{\large\bfseries \Themetext{#1}}
}

% Math
\newcommand{\Ceq}[2]{\stackrel{\text{\tiny{#1}}}{#2}} % use for commented =, \leq \geq etc.
\newcommand{\R}{\mathbb{R}}
\newcommand{\Z}{\mathbb{Z}}
\newcommand{\Mod}{\;\text{\bf mod}\;}
\newcommand{\Modulo}[1]{\;(\text{mod }#1)}
\newcommand{\Signum}[1]{\text{sgn}(#1)}
\newcommand{\Dul}[1]{\underline{\underline{#1}}}


% Få fancyhdr til å holde kjeft
\setlength{\headheight}{14pt} 

% Få pgfplots til å holde kjeft
\pgfplotsset{compat=1.15}

\lstdefinelanguage{Lie}
{
  % list of keywords
  morekeywords={
    include,
    if,
    let,
    fn,
    lambda,
    match,
    unless,
  },
  sensitive=true,       % keywords are case-sensitive
  morecomment=[l]{;},   % l is for line comment
  morestring=[b]"       % double quoted strings
}

\lstdefinestyle{themecode}{
    keywordstyle=\bfseries\Themetext,
    commentstyle=\textcolor{Gray}
}

\lstset{
    numbers=left,
    basicstyle=\ttfamily,
    style=themecode,
    extendedchars=true,
    showstringspaces=false,
    literate={λ}{{$\lambda$}}1
             {π}{{$\pi$}}1
             {->}{{$\rightarrow$}}2,
}


\begin{document}
% Front Page:
{
  \raggedleft
  {\Large Steffen Haug} \\
  {\large \bfseries \underline{Rettast}}\\[0.167\textheight]
  {\HUGE \bfseries \Themetext{Øving 8}} \\[\baselineskip]
  {\LARGE Diskret Matematikk}

}
\thispagestyle{empty}
\pagestyle{empty}
\newpage

\section*{Oppgåver til seksjon 9.1} % 7 40a 40c
\MajorTask{7}
\noindent Betraktar relasjonar på \(\Z\)

\Task{a} \(R = \{(x, y) \:|\: x \neq y\}\)

\noindent Relasjonen er ikkje refleksiv. Den inneheld alle \((x,y)\), sett vekk frå
\((a,a), \; a\in\Z\). Relasjonen er symmetrisk, fordi \(x \neq y \rightarrow y \neq x\).
Relasjonen er ikkje transitiv: til dømes er \((1,2), (2,1) \in R\) men \((1,1) \not\in R\).

\Task{b} \(R = \{(x,y) \:|\: xy \geq 1\}\)

\noindent Relasjonen er ikkje refleksiv fordi \((0,0)\not\in R\) sjølv om \(0 \in \Z\).
Relasjonen er symmetrisk fordi multiplikasjon er kommutativ. Relasjonen er transitiv:
\((a_1,a_2), (b_1,b_2) \in R \rightarrow a_1,b_2 \neq 0 \rightarrow (a_1, b_2) \in R\)

\Task{c} \(R = \{(x,y) \:|\: x = y + 1\}\)

\noindent Relasjonen er ikkje refleksiv fordi \(x = y + 1 \rightarrow x \neq y\). Relasjonen
er antisymmetrisk fordi \(x = y + 1 \rightarrow (p+1, p) \in R \text{, men } y = x - 1
\rightarrow (p, p+1) \not\in R\). Relasjonen er ikkje transitiv fordi
\((y+1, y), (y, y-1) \in R \rightarrow (y+1, y-1) \not\in R\)


\Task{d} \(R = \{(x,y) \:|\: x  \equiv y \Modulo{7}\}\)

\noindent Relasjonen er refleksiv fordi alle tal er ekvivalent med seg sjølv, i kva
modulo som helst. Relasjonen er symmetrisk, fordi \(x \equiv y \Modulo{7} \rightarrow
y \equiv x \Modulo{7}\). Relasjonen er transitiv fordi \(x \equiv y \Modulo{7},\:
y \equiv z \Modulo{7} \rightarrow x \equiv z \Modulo{7}\)

\Task{e} \(R = \{(x,y) \:|\: x \text{ er multiplum av } y\} = \{(x,y) \:|\: y | x\}\)

\noindent Relasjonen er refleksiv, fordi alle tal deler seg sjølv. Relasjonen er antisymmetrisk,
til dømes: \((15, 3) \in R\), men \((3, 15) \not\in R\). Relasjonen er transitiv, fordi
\(\forall a,b,c \; (a,b) \in R, (b,c) \in R \rightarrow b|a, c|b \rightarrow c|a \rightarrow
(a,c) \in R\)

\Task{f} \(\begin{array}{@{}rl}
             R =&\!\!\!\! \{(x,y) \:|\: x, y \text{ begge negative eller begge ikkje-negative}\} \\
             =&\!\!\!\! \{(x,y) \:|\: \Signum{x} = \Signum{y}\} \;\bm*
           \end{array}
\)

\noindent Relasjonen er refleksiv, fordi alle tal har same forteikn som seg sjølv.
Relasjonen er symmetrisk, fordi \((a,b) \in R \rightarrow \text{signum}(a)
= \text{signum}(b) \rightarrow (b, a) \in R\).
Relasjonen er transitiv, fordi \(\Signum{a} = \Signum{b}, \Signum{b} = \Signum{c}
\rightarrow \Signum{a} = \Signum{c} \rightarrow (a,c) \in R\)
\begin{align*}
  \bm* \; \Signum{x} :=
  \left\{\begin{array}{rl}
    -1 & \text{ dersom } x < 0 \\
    1 & \text{ elles}
  \end{array}\right.
\end{align*}
           
\Task{g} \(R = \{(x,y) \:|\: x = y^2\}\)

\noindent Relasjonen er ikkje refleksiv, fordi generelt er \(a \neq a^2\).
Relasjonen er ikkje symmetrisk, fordi \((x, y) \in R \rightarrow y = \sqrt{x} \rightarrow
(y, x) \not\in R\).
Relasjonen er ikkje transitiv, fordi \((a, b) \in R, (b, c) \in R \rightarrow a = c^4
\rightarrow (a,c) \not\in R\)

\newpage
\Task{h} \(R = \{(x,y) \:|\: x \geq y^2\}\)
           
\noindent Relasjonen er ikkje refleksiv, fordi generelt er \(a \not\geq a^2\).
Relasjonen er ikkje symmetrisk, fordi \((x,y) \in R \rightarrow x \geq y^2
\Ceq{$\bm*$}{\rightarrow} \sqrt{x} \geq y \rightarrow y \not \geq x^2
\rightarrow (y,x) \not\in R\)
Relasjonen er transitiv, fordi \((a,b) \in R, (b,c) \in R \Ceq{$\bm*$}{\rightarrow}
a \geq b \geq c \rightarrow (a,c) \in R\)

\noindent \(\bm*\) \(x \geq y^2 \rightarrow x \geq 0\)

\MajorTask{40}
\noindent La \(R_1\) og \(R_2\) vera ``deler''-relasjonen og ``er multilum av''-relasjonen
på \(\Z^+\). Det vil seie
\begin{align*}
  R_1 &= \{(a,b) \:|\: a|b\} \qquad R_2 = \{(a,b) \:|\: b|a\}
\end{align*}
Skal finne

\Task{a} \(R_1 \cup R_2 = \{(a, b) \;|\; a|b \lor b|a\}\)

\Task{c} \(R_1 \setminus R_2 = R_1 \setminus (R_1 \cap R_2) = \{(a,b) \;|\; a|b \land a \neq b\}\)

\noindent På grunn av at \(R_1 \cap R_2\) gjev \(a|b \land b|a \rightarrow b=ma \land a=nb \rightarrow a = n(ma)
\rightarrow n = m \rightarrow a = b\)

\section*{Oppgåver til seksjon 9.3}
\MajorTask{10}
\noindent Skal finne talet, heretter omtalt som \(T(M_R)\), på ikkje-null element
i matrisa som representerer \(R\)
på \(A = \{1,2,3,\dots 1000\}\), dei første tusen positive heiltala, dersom

\Task{a} \(R = \{(a,b) \mid a \leq b\}\)

\noindent
\begin{minipage}[]{0.5\textwidth}
  \begin{flalign*}
    M_R &= \begin{blockarray}{cccccccc}
      & & & & & & & \text{\bf sum} \\
      \begin{block}{[ccccccc]c}
        \TT 1 & \TT 1 & \TT 1 & \cdots & \TT 1 & \TT 1 & \TT 1 & 1000 \\
        0 & \TT 1 & \TT 1 & \cdots & \TT 1 & \TT 1 & \TT 1 & 999 \\
        0 & 0 & \TT 1 & \cdots & \TT 1 & \TT 1 & \TT 1 & 998 \\
        \vdots &\vdots&\vdots& \ddots & \vdots &\vdots & \vdots & \vdots \\
        0 & 0 & 0 & \cdots & \TT 1 & \TT 1 & \TT 1 & 3 \\
        0 & 0 & 0 & \cdots & 0 & \TT 1 & \TT 1 & 2 \\
        0 & 0 & 0 & \cdots & 0 & 0 & \TT 1 & 1 \\
      \end{block}
    \end{blockarray}&
  \end{flalign*}
\end{minipage}
\noindent\begin{minipage}[]{0.5\textwidth}
  Vi ser at \(T(M_R)\) er summen av heiltal frå 1 til 1000.
  \begin{align*}
    T(M_R) = \sum_{i=1}^{1000}i = \frac{1000^2 + 1000}{2} = \Dul{500500}
  \end{align*}
\end{minipage}


\newpage
\Task{b} \(R = \{(a,b) \mid a = b \pm 1\}\)

\noindent
\begin{minipage}{0.5\textwidth}
  \begin{flalign*}
    M_R &= \begin{blockarray}{cccccccc}
      & & & & & & & \text{\bf sum} \\
      \begin{block}{[ccccccc]c}
        0 & \TT 1 & 0 & \cdots & 0 & 0 & 0 & 1 \\
        \TT 1 & 0 & \TT 1 & \cdots & 0 & 0 & 0 & 2 \\
        0 & \TT 1 & 0 & \cdots & 0 & 0 & 0 & 2 \\
        \vdots &\vdots&\vdots& \ddots & \vdots &\vdots & \vdots & \vdots \\
        0 & 0 & 0 & \cdots & 0 & \TT 1 & 0 & 2 \\
        0 & 0 & 0 & \cdots & \TT 1 & 0 & \TT 1 & 2 \\
        0 & 0 & 0 & \cdots & 0 & \TT 1 & 0 & 1 \\
      \end{block}
    \end{blockarray}&
  \end{flalign*}
\end{minipage}
\begin{minipage}{0.5\textwidth}
  Vi ser at matrisa har 1 langs to ``diagonalar'' langs hoveddiagonalen,
  kvar med 999 einarar.
  \begin{align*}
    T(M_R) = 2 \cdot 999 = 1998
  \end{align*}
\end{minipage}


\Task{c} \(R = \{(a,b) \mid a + b = 1000\}\)

\noindent
\begin{minipage}{0.5\textwidth}
  \begin{flalign*}
    M_R &= \begin{blockarray}{cccccccc}
      & & & & & & & \text{\bf sum} \\
      \begin{block}{[ccccccc]c}
        0 & 0 & 0 & \cdots & 0 & \TT 1 & 0 & 1 \\
        0 & 0 & 0 & \cdots & \TT 1 & 0 & 0 & 1 \\
        0 & 0 & 0 & \cdots & 0 & 0 & 0 &  1 \\
        \vdots &\vdots&\vdots& \ddots & \vdots &\vdots & \vdots & \vdots \\
        0 & \TT 1 & 0 & \cdots & 0 & 0 & 0 & 1 \\
        \TT 1 & 0 & 0 & \cdots & 0 & 0 & 0 & 1 \\
        0 & 0 & 0 & \cdots & 0 & 0 & 0 & 0 \\
      \end{block}
    \end{blockarray}&
  \end{flalign*}
\end{minipage}
\begin{minipage}{0.5\textwidth}
  Matrisa har berre 1 langs ein diagonal, som i {\tt \TT b} har denne 999 einarar.
  \begin{align*}
    T(M_R) = 999
  \end{align*}
\end{minipage}


\Task{d} \(R = \{(a,b) \mid a + b \leq 1001\}\)

\noindent
\begin{minipage}{0.5\textwidth}
  \begin{flalign*}
    M_R &= \begin{blockarray}{cccccccc}
      & & & & & & & \text{\bf sum} \\
      \begin{block}{[ccccccc]c}
        \TT 1 & \TT 1 & \TT 1 & \cdots & \TT 1 & \TT 1 & \TT 1 & 1000 \\
        \TT 1 & \TT 1 & \TT 1 & \cdots & \TT 1 & \TT 1 & 0 & 999 \\
        \TT 1 & \TT 1 & \TT 1 & \cdots & \TT 1 & 0 & 0 & 998 \\
        \vdots &\vdots&\vdots& \ddots & \vdots &\vdots & \vdots & \vdots \\
        \TT 1 & \TT 1 & \TT 1 & \cdots & 0 & 0 & 0 & 3 \\
        \TT 1 & \TT 1 & 0 & \cdots & 0 & 0 & 0 & 2 \\
        \TT 1 & 0 & 0 & \cdots & 0 & 0 & 0 & 1 \\
      \end{block}
    \end{blockarray}&
  \end{flalign*}
\end{minipage}
\begin{minipage}{0.5\textwidth}
  \begin{align*}
    T(M_R) = \sum_{i=1}^{1000}i = \frac{1000^2 + 1000}{2} = \Dul{500500}
  \end{align*}
\end{minipage}


\newpage
\Task{e} \(R = \{(a,b) \mid a \neq 0\}\)

\noindent
\begin{minipage}{0.5\textwidth}
  \begin{flalign*}
    M_R &= \begin{blockarray}{cccccccc}
      & & & & & & & \text{\bf sum} \\
      \begin{block}{[ccccccc]c}
        \TT 1 & \TT 1 & \TT 1 & \cdots & \TT 1 & \TT 1 & \TT 1 & 1000 \\
        \TT 1 & \TT 1 & \TT 1 & \cdots & \TT 1 & \TT 1 & \TT 1 & 1000 \\
        \TT 1 & \TT 1 & \TT 1 & \cdots & \TT 1 & \TT 1 & \TT 1 & 1000 \\
        \vdots& \vdots& \vdots& \ddots & \vdots&\vdots & \vdots& \vdots \\
        \TT 1 & \TT 1 & \TT 1 & \cdots & \TT 1 & \TT 1 & \TT 1 & 1000 \\
        \TT 1 & \TT 1 & \TT 1 & \cdots & \TT 1 & \TT 1 & \TT 1 & 1000 \\
        \TT 1 & \TT 1 & \TT 1 & \cdots & \TT 1 & \TT 1 & \TT 1 & 1000 \\
      \end{block}
    \end{blockarray}&
  \end{flalign*}
\end{minipage}
\begin{minipage}{0.5\textwidth}
  \(0 \not\in A\), så \(a \neq 0\) er sant for alle relasjonar på A.
  Dermed er alle element i \(M\) lik 1.
  \begin{align*}
    T(M_R) = 1000^2 = 10^6
  \end{align*}
\end{minipage}


\MajorTask{14}
\noindent La \(R_1\) og  \(R_2\) vere relasjonar på mengda \(A\) representert ved
matrisene
\begin{align*}
  M_{R_1} = \begin{bmatrix}
    0 & 1 & 0 \\
    1 & 1 & 1 \\
    1 & 0 & 0
  \end{bmatrix}
  \qquad
  M_{R_2} = \begin{bmatrix}
    0 & 1 & 0 \\
    0 & 1 & 1 \\
    1 & 1 & 1
  \end{bmatrix}
\end{align*}
Skal finne matrisene som representerer

\Task{a} \(M_{R_1 \cup R_2} = \begin{bmatrix}
  0 & 1 & 0 \\ 1 & 1 & 1 \\ 1 & 1 & 1
\end{bmatrix}\)


\Task{b} \(M_{R_1 \cap R_2} = \begin{bmatrix}
  0 & 1 & 0 \\ 0 & 1 & 1 \\ 1 & 0 & 0
\end{bmatrix}\)


\Task{c} \(M_{R_2 \circ R_1} = \begin{bmatrix}
    0 & 1 & 0 \\
    0 & 1 & 1 \\
    1 & 1 & 1
  \end{bmatrix} \odot \begin{bmatrix}
    0 & 1 & 0 \\
    1 & 1 & 1 \\
    1 & 0 & 0
  \end{bmatrix} = \begin{bmatrix}
    1 & 1 & 1 \\
    1 & 1 & 1 \\
    1 & 1 & 1
\end{bmatrix}\)


\newpage
\section*{Oppgåver til seksjon 9.4}
\MajorTask{16}
\noindent Skal avgjere om sekvensane er stiar i grafen.

\Task{a} \(a,b,c,e\)

\noindent Ja


\Task{b} \(b,e,c,b,e\)

\noindent Nei, \(e \rightarrow c\) er ikkje gyldig


\Task{c}  \(a,a,b,e,d,e\)

\noindent Ja


\Task{d} \(b,c,e,d,a,a,b\)

\noindent Nei, \(d \rightarrow a\) er ikkje gyldig


\Task{e} \(b,c,c,b,e,d,e,d\)

\noindent Ja


\Task{f} \(a,a,b,b,c,c,b,e,d\)

\noindent Nei, \(b \rightarrow b\) er ikkje gyldig


\MajorTask{20}
\noindent La \(R\) vera relasjonen som inneheld \((a,b)\) dersom \(a\)
og \(b\) er byar slik at der er flyruter utan mellomlanding frå \(a\) til
\(b\). Skal avgjera når \((a,b)\) er i

\Task{a} $R^2$

\noindent Når byane er knytta med nøyaktig ei mellomlanding


\Task{a} $R^3$

\noindent Når byane er knytta med nøyaktig to mellomlandingar


\Task{a} $R^*$

\noindent Når byane er knytta gjennom ein sekvens av byar slik at kvar av dei
har ei flyrute til den neste.


\MajorTask{24}
\noindent Anta releasjonen \(R\) ikkje er refleksiv. Er relasjonen \(R^2\)
nødvendigvis òg ikkje-refleksiv?

\vspace{2mm}\noindent
Viser med moteksempel. Anta \(M_R\) er følgande matrise, og betraktar det boolske
matriseproduktet \(M_R^{[2]}\)
\begin{align*}
  \begin{bmatrix}
    1 & 1 & 1 \\
    1 & 0 & 0 \\
    1 & 0 & 0
  \end{bmatrix}
  \odot \begin{bmatrix}
    1 & 1 & 1 \\
    1 & 0 & 0 \\
    1 & 0 & 0
  \end{bmatrix}
  = \begin{bmatrix}
    1 & 1 & 1 \\
    1 & 1 & 1 \\
    1 & 1 & 1
  \end{bmatrix}
\end{align*}
Matriseproduktet viser at \(R^2\) kan vera refleksiv, sjølv om \(R\) ikkje er det.


\newpage
\section*{Oppgåver til seksjon 9.5}
\MajorTask{9}
\noindent Gjeve at \(A\) er ei ikkje-tom mengd, og \(f\) er ein funksjon med
\(A\) som definisjonsmengd. La \(R\) vera relasjonen på \(A\) slik at
\(R = \{(x,y) \mid f(x) = f(y)\}\).

\Task{a} Skal vise at \(R\) er ein ekvivalensrelasjon

\noindent {\bf Refleksivitet:} \(\forall a \in A\; f(a) = f(a) \implies R\) er refleksiv

\noindent {\bf Symmetri:} \((a,b) \in R \implies f(a) = f(b) \implies (b,a) \in R
\rightarrow R\) er symmetrisk

\noindent {\bf Transitivitet:} \((a,b) \in R, (b,c) \in R \implies f(a) = f(b) = f(c)
\implies (a,c) \in R \implies R\) er transitiv


\Task{b} Kva er ekvivalensklassene til \(R\)?
\noindent Sidan \(a \thicksim b\) med omsyn til \(R\) berre når \(f(a) = f(b)\) er
ekvivalensklassene \([a] = \{a \mid f(a) = k \in V_f\}\), altso mengdene med element
\( a\in A\) som har same verdi \(f(a)\).


\MajorTask{16}
\noindent La \(R\) vera relasjonen på settet med ordna par av positive heiltal slik at
\begin{align*}
  R = \{((a,b),(c,d)) \mid ad = bc\}
\end{align*}
Skal vise at \(R\) er ein ekvivalensrelasjon.

\noindent {\bf Refleksivitet:} \(\forall a,b \; (ab = ab) \implies ((a,b),(a,b)) \in R
\implies R\) er refleksiv.

\noindent {\bf Symmetri:} \(((a,b), (b,c)) \in R \implies ad=bc \implies cb=da
\implies ((c,d),(a,b)) \in R \implies R\) er symmetrisk.

\noindent {\bf Transitivitet:}
\begin{align*}
  &((a,b), (c,d)) \in R \land ((c,d), (e,f)) \in R \\
  \implies& ad = bc \land cf = de \\
  \implies& ad = b\frac{de}{f} \\
  \implies& af = be \\
  \implies& ((a,b),(e,f)) \in R \implies R \text{ er transitiv}
\end{align*}
Alle tre krava er oppfylde, og dermed er \(R\) ein ekvivalensrelasjon.





\end{document}
