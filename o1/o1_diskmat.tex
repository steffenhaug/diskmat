\documentclass[a4paper, 11pt]{article}
\usepackage[T1]{fontenc}
\usepackage[utf8]{inputenc}
\usepackage[norsk]{babel}
\usepackage{graphicx}
\usepackage[dvipsnames]{xcolor}
\usepackage[fleqn]{amsmath}
\usepackage{amsthm, amssymb}
\usepackage{microtype}
\usepackage{array}
\usepackage{caption}
\usepackage[norsk]{babel}
\usepackage[all]{nowidow}
%\usepackage{indentfirst}
\usepackage{mathtools}
\usepackage{sectsty}
\usepackage{tikz}
\usetikzlibrary{calc,arrows}
\usepackage{pdfpages}
\usepackage{amssymb}
\usepackage{bbold}
\usepackage{cancel}

% Unicode-inkompatibilitet mactex/texlive?
\DeclareUnicodeCharacter{00A0}{ }

% Få fancyhdr til å holde kjeft
\setlength{\headheight}{14pt} 

%-------------------------------------%
% Dokument-settings
%-------------------------------------%
\newcommand{\tittel}{Øving 1}
\newcommand{\fag}{Diskret Matematikk}
\newcommand{\fagkode}{TMA4140}
\newcommand{\forfatter}{Steffen Haug}


\renewcommand{\qedsymbol}{$\themecolor{\blacksquare}$}
\newcommand{\lheq}{\stackrel{\text{\tiny{L'Hop}}}{=}}
\newcommand{\ceq}[2]{\stackrel{\text{\tiny{#1}}}{#2}}
\newcommand{\R}{\mathbb{R}}
\newcommand{\vect}[1]{\mathbf{#1}}
\newcommand{\deloppg}[1]{\vspace{1mm}\noindent \textbf{\themecolor{#1:}}}
\newcommand{\steg}[1]{\vspace{1mm}\noindent {\themecolor{#1:}}}
\newcommand{\dint}{\int\displaylimits}

\newcommand{\cvec}[1]{\begin{bmatrix}#1\end{bmatrix}}

\newcommand{\themeshade}{Mahogany}
\newcommand{\themecolor}[1]{\textcolor{\themeshade}{#1}}
\sectionfont{\color{\themeshade}}

\DeclarePairedDelimiter\abs{\lvert}{\rvert}
\def\dul#1{\underline{\underline{#1}}}
\def\multipart#1{
\left\{
	\begin{array}{ll}
		#1
	\end{array}
\right.
}


\usepackage{fancyhdr}
\lhead{\tittel \:{\color{\themeshade}\fagkode}}
\rhead{\forfatter}

\author{\forfatter}
\date{}

\newcommand*{\titleTH}{\begingroup % Create the command for including the title page in the document
\raggedleft % Right-align all text
\vspace*{\baselineskip} % Whitespace at the top of the page

{\Large \forfatter}\\[0.167\textheight] % Author name

{\LARGE\bfseries \tittel}\\[\baselineskip] % First part of the title, if it is unimportant consider making the font size smaller to accentuate the main title

{\themecolor{\Huge \fag}}\\[\baselineskip] % Main title which draws the focus of the reader

{\Large \textit{\fagkode}}\par % Tagline or further description

\vfill % Whitespace between the title block and the publisher
\endgroup}


\begin{document}

\pagestyle{empty}
\titleTH
\newpage\pagestyle{fancy}

\section*{Oppgåver til seksjon 1.1} % 12c 12f 14a 14e

\subsection*{Oppgåve 12cf}
La $p, q$ og $r$ vera proposisjonane
\begin{align*}
    p&: \text{You have the flu.} \\
    q&: \text{You miss the final examination.} \\
    r&: \text{You pass the course.}
\end{align*}

\noindent Proposisjonar uttrykte som engelske setningar:
\begin{align*}
    q \rightarrow \neg r: &\;\text{If you have the flu, you will not pass the course.} \\
    (p \land q) \lor (\neg q \land r): &\;\text{You have the flu and you miss the final examination,}\\ 
    &\;\text{or you do not miss the final examination}\\
    &\;\text{and you pass the course.} \\
\end{align*}

\subsection*{Oppgåve 14}
La $p,q$ og $r$ vera proposisjonane
\begin{align*}
    p&: \text{You get an A on the final exam.}\\
    q&: \text{You do every exercise in this book.}\\
    r&: \text{You get an A in this class.}\\
\end{align*}

\noindent \themecolor{Proposisjonar uttrykte som logiske uttrykk:}

\deloppg{a} You get an A in this class, but you do not do every exercise in this book.
\[
    r \land \neg q    
\]

\deloppg{e} Getting an A on the final and doing every exercise in this book
is sufficient for getting an A in this class.
\[
    (p \land q) \rightarrow r
\]


\newpage \section*{Oppgåver til seksjon 1.3}

\subsection*{Oppgåve 10}
Skal vise at uttrykka er tautologiar.

\deloppg{a} $[\neg p \land (p \lor q)] \rightarrow q$

\begin{tabular}{c c c c c | c}
    $p$ &  $q$  &  $\neg p$ & $a: p \lor q$ & $b: \neg p \land a$   & $b \rightarrow q$ \\
    \hline
    0   & 0     & 1         & 0             & 0                     & 1                 \\
    0   & 1     & 1         & 1             & 1                     & 1                 \\
    1   & 0     & 0         & 1             & 0                     & 1                 \\
    1   & 1     & 0         & 1             & 0                     & 1                 \\
\end{tabular}

\deloppg{b} $[(p \rightarrow q) \land (q \rightarrow r)] \rightarrow (p \rightarrow r)$

\begin{tabular}{ccccccc|c}
    $p$ & $q$ & $r$ & $a: p \rightarrow q$ & $b: q \rightarrow r$
    & $c: a \land b$ & $d: p \rightarrow r$ & $c \rightarrow d$\\
    \hline
    0 & 0 & 0 & 1 & 1 & 1 & 1 & 1 \\
    0 & 0 & 1 & 1 & 1 & 1 & 1 & 1 \\
    0 & 1 & 0 & 1 & 0 & 0 & 1 & 1 \\
    0 & 1 & 1 & 1 & 1 & 1 & 1 & 1 \\
    1 & 0 & 0 & 0 & 1 & 0 & 0 & 1 \\
    1 & 0 & 1 & 0 & 1 & 0 & 1 & 1 \\
    1 & 1 & 0 & 1 & 0 & 0 & 0 & 1 \\
    1 & 1 & 1 & 1 & 1 & 1 & 1 & 1 \\
\end{tabular}

\deloppg{c} $[p \land (p \rightarrow q)] \rightarrow q$

\begin{tabular}{cccc|c}
    $p$ & $q$ & $a: p \rightarrow q$ & $b: p \land a$ & $b \rightarrow q$ \\
    \hline
    0   & 0   & 1                    & 0              & 1 \\
    0   & 1   & 1                    & 0              & 1 \\
    1   & 0   & 0                    & 0              & 1 \\
    1   & 1   & 1                    & 1              & 1 \\
\end{tabular}

\deloppg{d} $[(p \land q) \land (p \rightarrow r) \land (q \rightarrow r)] \rightarrow r$

\begin{tabular}{ccccccc|c}
    $p$ & $q$ & $r$ & $a: p \land q$ & $b: p \rightarrow r$
    & $c: q \rightarrow r$ & $d: a\land b\land c$ & $d \rightarrow r$ \\
    \hline
    0 & 0 & 0 & 0 & 1 & 1 & 0 & 1 \\ 
    0 & 0 & 1 & 0 & 1 & 1 & 0 & 1 \\
    0 & 1 & 0 & 0 & 1 & 0 & 0 & 1 \\
    0 & 1 & 1 & 0 & 1 & 1 & 0 & 1 \\
    1 & 0 & 0 & 0 & 0 & 1 & 0 & 1 \\
    1 & 0 & 1 & 0 & 1 & 1 & 0 & 1 \\
    1 & 1 & 0 & 1 & 0 & 0 & 0 & 1 \\
    1 & 1 & 1 & 1 & 1 & 1 & 1 & 1 \\
\end{tabular}


\newpage
\section*{Oppgåver til seksjon 1.4}

\subsection*{Oppgåve 24}

\deloppg{d} All students in your class can solve quadratic equations.
\begin{align*}
    &P(x): \text{student $x$ kan løyse kvadratiske likningar}\\
    &Q(x): \text{student $x$ er i klassa}
\end{align*}
\begin{align*}
    &\themecolor{x\in U = \{\text{alle elevar i klassa}\}} \\
    &\forall x P(x)\\
    &\themecolor{x\in U = \{ \text{alle mennesker} \}} \\
    &\forall x (Q(x)\rightarrow P(x))
\end{align*}


\deloppg{e} Some students in your class does not want to be rich.
\begin{align*}
    &P(x): \text{student $x$ har lyst å bli rik}\\
    &Q(x): \text{student $x$ er i klassa}
\end{align*}
\begin{align*}
    &\themecolor{x\in U = \{\text{alle elevar i klassa}\}} \\
    &\exists x P(x) \\
    &\themecolor{x\in U = \{ \text{alle mennesker} \}} \\
    &\exists x (Q(x)\rightarrow P(x))
\end{align*}


\section*{Oppgåver til seksjon 1.5}
 
\subsection*{Oppgåve 12}
\begin{align*}
    I(x)&: \text{$x$ has an Internet connection} \\
    C(x,y)&: \text{$x$ and $y$ have chatted over the Internet}\\
    x \in U &= \{ \text{elevar i klassa} \}
\end{align*}

\deloppg{b} Rachel has not chatted over the Internet with Chelsea.
\[
    \exists x \exists y \neg C(x,y)
\]
I klassa {\em eksisterer} ein person $x$, nemleg Rachel,
og der {\em eksisterer} ein person $y$, Chelsea,
slik at $C$ {\em ikkje} held.


\deloppg{e} Sanjay has chatted with everyone except Joseph.
\[
    \exists x \exists y \neg C(x,y)
\]
I klassa {\em eksisterer} ein person $x$, Sanjay,
og der {\em eksisterer} ein person $y$, Joseph,
slik at $C$ {\em ikkje} held.

\subsection*{Oppgåve 30}
\deloppg{c} $\neg \exists y (Q(y) \land \forall x \neg R(x,y))$
\begin{align*}
     &\; \neg \exists y (Q(y) \land \forall x \neg R(x,y)) \\
    =&\; \forall y \neg (Q(y) \land \forall x \neg R(x,y)) \\
    =&\; \forall y (\neg Q(y)) \lor (\neg \forall x \neg R(x,y)) \\
    =&\; \forall y \neg Q(y) \lor \exists x R(x,y)
\end{align*}

\deloppg{e} $\neg \exists y (\forall x \exists z T(x,y,z) \lor \exists x \forall z U(x,y,z))$
\begin{align*}
     &\; \neg \exists y (\forall x \exists z T(x,y,z) \lor \exists x \forall z U(x,y,z)) \\
    =&\; \forall y \neg (\forall x \exists z T(x,y,z) \lor \exists x \forall z U(x,y,z)) \\
    =&\; \forall y (\neg \forall x \exists z T(x,y,z)) 
        \land (\neg \exists x \forall z U(z,y,z)) \\
    =&\; \forall y (\exists x \forall z \neg T(x,y,z))
        \land (\forall x \exists z \neg U(x,y,z)) \\
\end{align*}

\end{document}

